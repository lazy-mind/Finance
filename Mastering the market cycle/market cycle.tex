\documentclass[10pt,portrait]{article}
\usepackage{amssymb,amsmath,amsthm,amsfonts}
\usepackage{multicol,multirow}
\usepackage{calc}
\usepackage{ifthen}
% \usepackage[a4paper, total={6in, 8in}]{geometry}
\usepackage[a4paper, portrait, margin=0.4in]{geometry}
\usepackage[colorlinks=true,citecolor=blue,linkcolor=blue]{hyperref}
\usepackage{graphicx}
\usepackage{cases}
\usepackage{indentfirst}

\makeatletter
\renewcommand{\section}{\@startsection{section}{1}{0mm}%
                                {-1ex plus -.5ex minus -.2ex}%
                                {0.5ex plus .2ex}%x
                                {\normalfont\large\bfseries}}
\renewcommand{\subsection}{\@startsection{subsection}{2}{0mm}%
                                {-1explus -.5ex minus -.2ex}%
                                {0.5ex plus .2ex}%
                                {\normalfont\normalsize\bfseries}}
\renewcommand{\subsubsection}{\@startsection{subsubsection}{3}{0mm}%
                                {-1ex plus -.5ex minus -.2ex}%
                                {1ex plus .2ex}%
                                {\normalfont\small\bfseries}}
\makeatother
\setcounter{secnumdepth}{0}





\graphicspath{ {./images/} }




\begin{document}


\newpage
    \begin{center}
         \Large{\textbf{Why do we study market cycle}} \\
    \end{center}

    \subsection{\color{blue}It is hard to know about the macro market}
    \begin{itemize}
        \item It is hard to know about hte ``macro future'' and that's why it is hard to get superior than other investors by being more knowledgable in the macro future. Insightful and valuable as macro future prediction is, this piece information does not serve the regular investors well. Like Warren Buffet said: ``a desirable piece of information should be important and knowable''. In this definition, macro future is less a desirable information simply because it is not that ``knowable''
    \end{itemize}

    \subsection{\color{blue}Then what are other ways to get insights in order to gain profit}
    \begin{itemize}
        \item More knowledge: trying to know more than others about what I call ``the knowable'', such as fundamentals
        \item Price and Value: being disciplined as to the appropriate price to pay for a participation in those fundamentals
        \item Know your environment: being disciplined as to the appropriate price to pay for a participation in those fundamentals
    \end{itemize}
    
        The first 2 altogether consist the realm of ``security analysis and value investing'', which is to take advantage of discrepancies betweeen price and value.

        The traditional view is to assemble a highest return-risk ratio portfolio and use leverage to invest in it. This might be true in the long run, but we want to have a way to have the kind the investmeent insight that will come into effect in a year or less

        So the 3rd idea comes into play, it is about positioning our portfolio in different points of time, decide being aggressive or defensive.

    \subsection{\color{blue}This means we have to master: ``tendency''}
    \begin{itemize}
        \item Tendency means that: we have an opinion of what is going to happen
        \item Tendency means that: we have a probalistic model on the likelihood of that happening
    \end{itemize}

        So is to say that ``tendency'' is about us understand the market and the world as a probalistic model. To come up with such model, we need to:

    \begin{itemize}
        \item understand nature and importance of cycle
        \item live through many cycles to learn lessons from them
        \item notes the pattern and understand the reason behind
        \item know what cycle tells us to act
    \end{itemize}


    \subsection{\color{blue}The things to look at in tendency are}
    \begin{itemize}
        \item Are we at beginning or the end: Warren Buffet says ``the wise do in the beginning, the fool do in the end''
        \item Does investors being driven by greed or by fear: this will result in a misprice in the market
        \item Does investors being risk-adverse or foolish risk-tolerant: this tells us how many opportunities are out there
        \item Should we act aggressive or defensive accordingly
    \end{itemize}

            % $
            % \begin{cases}
            %     \text{}\\
            %     \text{}\\
            %     \text{}\\
            %     \text{}\\
            % \end{cases}
            % $

            % \includegraphics[width=0.8\textwidth]{}



% \newpage
%     \begin{center}
%          \Large{\textbf{Why investment with machine learning is so hard}} \\
%     \end{center}


    % \subsection{}
    % \begin{itemize}
    %     \item 
        % \item 
        % \item 
        % \item 
        % \item 
        % \item 
        % \item 
    % \end{itemize}


            % $
            % \begin{cases}
            %     \text{}\\
            %     \text{}\\
            %     \text{}\\
            %     \text{}\\
            % \end{cases}
            % $

            % \includegraphics[width=0.8\textwidth]{}

\end{document}