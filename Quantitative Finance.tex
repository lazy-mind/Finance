% style setup
% -----------------------------------------------------------------------
\documentclass[10pt,landscape]{article}
\usepackage{amssymb,amsmath,amsthm,amsfonts}
\usepackage{amsmath}
\usepackage{amssymb}
\usepackage{multicol,multirow}
\usepackage[utf8]{inputenc}
\usepackage{calc}
\usepackage{ifthen}
\usepackage[landscape]{geometry}
\usepackage[colorlinks=true,citecolor=blue,linkcolor=blue]{hyperref}
\usepackage{graphicx}


\ifthenelse{\lengthtest { \paperwidth = 11in}}
    { \geometry{top=.5in,left=.5in,right=.5in,bottom=.5in} }
	{\ifthenelse{ \lengthtest{ \paperwidth = 297mm}}
		{\geometry{top=1cm,left=1cm,right=1cm,bottom=1cm} }
		{\geometry{top=1cm,left=1cm,right=1cm,bottom=1cm} }
	}
\pagestyle{empty}
\makeatletter
\renewcommand{\section}{\@startsection{section}{1}{0mm}%
                                {-1ex plus -.5ex minus -.2ex}%
                                {0.5ex plus .2ex}%x
                                {\normalfont\large\bfseries}}
\renewcommand{\subsection}{\@startsection{subsection}{2}{0mm}%
                                {-1explus -.5ex minus -.2ex}%
                                {0.5ex plus .2ex}%
                                {\normalfont\normalsize\bfseries}}
\renewcommand{\subsubsection}{\@startsection{subsubsection}{3}{0mm}%
                                {-1ex plus -.5ex minus -.2ex}%
                                {1ex plus .2ex}%
                                {\normalfont\small\bfseries}}
\makeatother
\setcounter{secnumdepth}{0}
\setlength{\parindent}{0pt}
\setlength{\parskip}{0pt plus 0.5ex}
% -----------------------------------------------------------------------
% -----------------------------------------------------------------------
% -----------------------------------------------------------------------

\title{Quick Guide to LaTeX}

\begin{document}

\footnotesize

% -----------------------------------------------------------------------
% -----------------------------------------------------------------------
% -----------------------------------------------------------------------



\begin{center}
     \Large{\textbf{Quantitative Finance -- Derivatives}} \\
\end{center}

\begin{multicols}{2}
\setlength{\premulticols}{1pt}
\setlength{\postmulticols}{1pt}
\setlength{\multicolsep}{1pt}
\setlength{\columnsep}{2pt}


\section{Time value of money}
\begin{itemize}
	\item money has time value, because we can reinvest them
	\item always picture a timeline in thinking about this problem
\end{itemize}

\section{Interest}
\begin{itemize}
	\item The easiest way to make money is by earning interest from the bank
	\item simple interest rate (return calculation): $a_t = a_0*(1+it)$
	\item compound interest rate (keep reinvest): $a_t = a_0(1+i)^t$
	\item effective interest rate during a period: $i = \frac{a_n - a_{n-1}}{a_{n-1}}$
	\item nominal rate of interest: $i^{(m)}$, meaning that this interest i is paid m times a year\\
		an amount of $\frac{i^{(m)}}{m}$ will be paid at each time interval, therefore at the end $=(1+\frac{i^{(m)}}{m} )^m$\\
		let m goes to infinity, we have the continuous form $(1+\frac{r}{m} )^{mt} \longrightarrow e^{rt}$
	
\end{itemize}

\section{Annuities and Perpetuities}
\begin{itemize}
	\item annuity: fixed payment at each time point\\
			annuity-immediate: payment at the end of interval\\
			annuity-due: payment at the beginning of interval
	\item perpetuities: fixed payment at each time point, until infinity\\
		perpetuity-immediate: payment at the end of interval\\
		perpetuity-due: payment at the beginning of interval
\end{itemize}

\section{Bonds: a fixed income instrument}
\begin{itemize}
	\item market graph of bonds: what kinds of bonds are out there
	\item zero coupon bond: no coupon payment, only pay face value at maturity (treat as discount)\\
			zero rate: the annualized rate of return (annual compound) $\longrightarrow B(t,T)(1+R(t,T))^{T-t} = 1$\\
			continuous zero rate: $\longrightarrow B(t,T)e^{(T-t)R(t,T)} = 1$   [treat R(t,T) as an interest rate]\\
			k-times per year: $\longrightarrow B_{(t,T)}(1+\frac{R_{(r,T)}}{m} )^m(T-t) = 1$
	\item Bond with coupon: pay at specific time\\
			
			Yield-to-Maturity: the interest rate that makes present value of cash flow equals bond price\\
			or we can say that the YTM is the IRR at this point: internal rate of return\\
			it tells us the interest rate that this investment is representing, so that we can compare\\
			
			Annual Coupon Bond: $P = \sum \frac{CF_t}{(1+y)^t}$\\
			Semi annual coupon bond: $P = \sum \frac{CF_t}{(1+\frac{y}{2	} )^{2t}}$\\
			
			So what happened in the market $\longrightarrow$ we calculate the bond price, by the risk-free rate\\
			Therefore it is easy to imagine: \\
			if risk-free rate goes up, the current bond price should drop\\
			if risk-free rate goes down, the current bond price should rise\\
			In short, the bond price is negatively correlated to the YTM\\
			
			It depends on how we structure the dependent variable and independent variables:\\
			1. Calculate the bond price: the DV is bond price, IDV is the interest rate, negative correlated.\\
			2, Calculate YTM: the DV is YTM, IDV is bond price, if bond price is high, meaning the YTM must be low\\
	\item Relationship between zero coupon bond and coupon bond:\\
			When a coupon bond pays no coupon, it is the same as zero coupon bond $\longrightarrow$ same price and YTM\\
			With coupon bond introduce coupon, we have 2 dimension to interpret the results:\\
			1. how the price will change: with introduction of future cash flow, interest rate does not change, price will rise; with price rise, the YTM will drop\\
			2. how the YTM will change: not a good direction, stick to PRICE $\longrightarrow$ YTM path
	\item Since the zero coupon bond has a higher YTM than coupon bond, it is some how confusing because we may think the coupon bond gives us more return.
	\item We introduce the "Par Yield" to do standardization on the returns:\\
	\item	Par Bond: the price = face value $\longrightarrow$ issue at par $\longrightarrow$ YTM = annual coupon rate\\
		Par Yield: when a bond is issue at par, the coupon rate (annualized return) is equal to YTM (calculated annualized return)
	\item Dollar Duration: the $\frac{dP}{dy}$ the derivatives of price w.r.t. YTM\\
			Modified Duration: $-\frac{1}{P} \frac{dp}{dy}$, Dur normalized by price\\
			Dollar Value of 1 basis point (DV01): $ = Dur \cdot (-1 bp)$\\
			
			At a specific price level, the Dollar Duration Dur, decrease as:\\
			1. maturity increase (Dur is a price sensitivity measure to yield, with T goes up, it is more sensitive)\\
			2. coupon increases (1. more money to discount, more sensitive. 2. more coupon means price up, then more sensitive)\\
			3. YTM decrease (look at the graph, when YTM is small, it is steeper)\\
	
\end{itemize}

\section{Forward Rates}
\begin{itemize}
	\item Diagram and calculation: separate the time into different intervals, make investment using respective rates
	\item 
\end{itemize}


\section{Futures}


\section{Options}
* Memoryless:\\

\section{Bonds}
* Memoryless:

\end{multicols}





% -----------------------------------------------------------------------
% -----------------------------------------------------------------------
% -----------------------------------------------------------------------





\newpage

%\includegraphics[width=60mm,scale=1]{images/PACF-2.png}
\begin{center}
     \Large{\textbf{Financial Assets: Derivatives}} \\
\end{center}\\

\begin{multicols}{2}
\setlength{\premulticols}{1pt}
\setlength{\postmulticols}{1pt}
\setlength{\multicolsep}{1pt}
\setlength{\columnsep}{2pt}


\section{Forwards / Futures}
\begin{itemize}
	\item Futures: Traded in the exchange, so basically standardized contract
	\item Forwards: customized
	\item Keywords: strike price, maturity day, long, short\\
	
	Agencies
	\item Clearing house: ensure trader fulfill obligations
	\item Margin: Trader deposit margin to fulfill obligations:\\
		Initial margin ; Maintenance margin ; Variation margin ;
	\item 
	
\end{itemize}

\section{Options}
\begin{itemize}
	\item Vanilla Option / European Option: fixed maturity date, strike price
	\item American Option: can exercise before maturity date
	\item Bermudan Option: can exercise on pre-determined date
	\item Barrier Option: can exercise once cross the barrier level: \\
			up and out, down and out, up and in, down and in\\
			1) Callable Bull: if always bull, then it is valid (never drop down barrier) [down and out]
			2) Callable Bear: if always bear, then it is valid (never go up barrier) [up and out]
			
	\item Asian Option: the payoff determined by average price of pre-set period of time\\
	
	Option strategies: Notice difference between prices and payoff
	\item Straddle: long 1 call and 1 put with same K and T
	\item Strangle: long call and put with same K and T
	\item Bull Spread: make money when price up: Buy call with $K_1$, sell call with $K_2$
	\item Bear Spread: make money when price down: Buy call with $K_1$, sell call with $K_2$
	\item Butterfly: long 2 calls, short call * 2
	
\end{itemize}

\section{Warrents}
\begin{itemize}
	\item equity warrants: issue by company. represent the right to subscribe to equity securities
	\item derivative warrants: issued by 3rd party.
\end{itemize}

\section{Swap: 2 parties exchange financial instruments}
\begin{itemize}
	\item Interest rate swap\\
			A: expect Libor drops ; B: expect Libor ups\\
			????
	\item Benefit: 1) hedge against interest rate exposure ; 2) leverage comparative advantage\\
			A: AAA rated company ; B: BBB rated company.\\
			Company A will definitely have less lending interest rate in the market, in all loans\\
			But that doesn't mean company A can't do better. We can calculate the spread and see what is company A's comparative advantage, and borrow only that kind of loan, other company can use their comparative advantage P38
	\item 
\end{itemize}


\end{multicols}





% -----------------------------------------------------------------------
% -----------------------------------------------------------------------
% -----------------------------------------------------------------------


\newpage

\begin{center}
     \Large{\textbf{Financial Risk: Value at Risk}} \\
\end{center}\\

\begin{multicols}{2}
\setlength{\premulticols}{1pt}
\setlength{\postmulticols}{1pt}
\setlength{\multicolsep}{1pt}
\setlength{\columnsep}{2pt}

\section{Return}
\begin{itemize}
	\item Arithmetic Return: $R_t = \frac{S_t - S_{t-1}}{S_{t-1}}$
	\item Log Return: $log(1+R_t) = log(\frac{S_t}{S_{t-1}}) = log(S_t) - log(S_{t-1})$
\end{itemize}

\section{Volatility Measure}
\begin{itemize}
	\item $\sigma_s = \sqrt{\frac{1}{n-1} \sum (\mu_i - \bar \mu)^2}$
	\item Then the volatility over N periods: $\sqrt{N}\sigma_s$
\end{itemize}

\section{VaR: Value-at-Risk}
\begin{itemize}
	\item if there is a distribution graph, starting from left to some point, this region is $\alpha$
	\item the rest region to the right is $1-\alpha$, stands for the confidence: 95\%, 99\% etc.
	\item We called $\alpha$ as the "tail probability", since it is the worst scenario
	\item In a graphical sense, VaR is the $p-th$ quantile of the PnL distribution
	\item "Relative VaR": $E[PnL] - VaR$
\end{itemize}

\section{Modeling the PnL distribution}
\begin{itemize}
	\item The fundamental of finding VaR is to model the portfolio PnL distribution
	\item Parametric Approach:\\
			Keypoint: construct the mean $\mu$ and volatility $\sigma$\\
			Approach: convert to approach "standard normal distribution"\\
			Then we can calculate VaR ; or extend this to a multi-time period scenario\\
	\item Non-Parametric Approach: Historical Simulation\\
			1. we want to know the price at tomorrow\\
			2. we convert the a problem: how much will the portfolio value change\\
			3. based on past period-wise change data, we treat it as distribution\\
			4. select the change we need\\
\end{itemize}
			
\end{multicols}


\begin{center}
     \Large{\textbf{Financial Risk: Hedging, Optimal Hedging}} \\
\end{center}\\

\begin{multicols}{2}
\setlength{\premulticols}{1pt}
\setlength{\postmulticols}{1pt}
\setlength{\multicolsep}{1pt}
\setlength{\columnsep}{2pt}

\section{Idea}
\begin{itemize}
	\item Hedge against a instrument that is negatively correlated to your portfolio
\end{itemize}

\section{Quantify}
\begin{itemize}
	\item we want to minimize variance of new portfolio : $S+NF$, with $S$ original portfolio
	\item we first construct the variance: $\sigma_s + N^2 \sigma_f + 2N\sigma_{s,f}$
	\item Then N can be solved by using the solution for a square equation
\end{itemize}

\end{multicols}

% -----------------------------------------------------------------------
% -----------------------------------------------------------------------
% -----------------------------------------------------------------------





\newpage

\begin{center}
     \Large{\textbf{Coursera - Columbia - No Arbitrage}} \\
\end{center}\\

\begin{multicols}{2}
\setlength{\premulticols}{1pt}
\setlength{\postmulticols}{1pt}
\setlength{\multicolsep}{1pt}
\setlength{\columnsep}{2pt}

\section{Example of using No-Arbitrage pricing: }
\begin{itemize}
	\item \text{\color{red}1. A bond that pays A in 1 year:}
		\\Pricing 1: portfolio that: buy 1 bond, borrow $\frac{A}{1+r}$
		\\Pricing 2: portfolio that: sell 1 bond, lend $\frac{A}{1+r}$
	\item \text{\color{red}2. Use compounding interest rate for pricing:}
	\item \text{\color{red}3. Floating Rate Bond:}
		\\The price of a floating rate bond, is equal to, the face value.
	\item 
\end{itemize}


			
\end{multicols}


\begin{center}
     \Large{\textbf{Cousera - Columbia - No Arbitrage in various assets}} \\
\end{center}\\

\begin{multicols}{2}
\setlength{\premulticols}{1pt}
\setlength{\postmulticols}{1pt}
\setlength{\multicolsep}{1pt}
\setlength{\columnsep}{2pt}

\section{Swaps}
\begin{itemize}
	\item What is it: a trade between fixed and float rates
	\item Why is it: leverage relative strength, capture trends in other markets
	\item How does it works: do your best, borrow your weakness
		\\1. A has relative strength in fixed rate
		\\2. B has relative strength in floating rate
		\\3. A will borrow fixed rate, B will borrow floating rate
		\\4. A borrow floating from B, pays fixed to B 
	\item How to price it: What should be the exchange rate pay to each other
		\\1. Swap should be of 0 value to each party
		\\2. Write the Cash Flow PV for one side
		\\3. Set it to 0 and we calculate the payment rate
\end{itemize}

\section{Futures}
\begin{itemize}
	\item What is it: An traded an exchange standardized contract
	\item Why need it: standardized, counter party risk, solved multiple price for same maturity
	\item How does it works: Margin call (go to initial margin), clearing house
	\item How to price it: 
		\\
	\item Minimum variance hedging (since perfect hedge is impossible):
		\\
\end{itemize}

\section{Options}
\begin{itemize}
	\item In the money, at the money, out of the money
	\item Put-Call Parity: call + cash = put + stock
		\\ try to buy a stock, and use money to help me
		\\ try to sell a stock, and hold a stock to help me
		\\ $C + Ke^{-rt} = P + S$
		\\ $C + Ke^{-rt} = P + S - D$ (with dividend)
	\item Bounds of option prices
\end{itemize}

\section{Binomial Tree Pricing}
\begin{itemize}
	\item 1-period Binomial Tree: price goes up and down with probabilities, allow short sell
	\item Multi-period Binomial Tree: extend with same prob and up/down scale
	\item Replications: (buying shares, investing in cash) $\leftrightarrow$ (option)
		\\ make the payoff equals
	\item European and American Pricing: use risk-neutral prob to price:
		\\ $q = \frac{R-d}{u-d}$. $C_0 = \frac{1}{R} [qC_u + (1-q)C_d ]$
\end{itemize}


\end{multicols}

% -----------------------------------------------------------------------
% -----------------------------------------------------------------------
% -----------------------------------------------------------------------










% style end
% -----------------------------------------------------------------------
\end{document}
% -----------------------------------------------------------------------
